\documentclass[12pt,addpoints,answers]{exam}
\usepackage[utf8]{inputenc}
\usepackage{amsmath,amsfonts,amssymb,amsthm}
\usepackage[margin=1in]{geometry}
\usepackage{mathtools}
\usepackage{hyperref}
\usepackage{fullpage}
\usepackage{microtype}
\usepackage{xspace}
\usepackage[svgnames]{xcolor}
\usepackage[sc]{mathpazo}
\usepackage{enumitem}
\usepackage{bm}

\pagestyle{head}

% TIRET DEFINITIONS

\newlength{\saveparindent}
\setlength{\saveparindent}{\parindent}
\newlength{\saveparskip}
\setlength{\saveparskip}{\parskip}
\newcounter{ctr}
\newcounter{savectr}
\newcounter{ectr}
\newcounter{sctr}


\newenvironment{tiret}{%
\begin{list}{\hspace{2pt}\rule[0.5ex]{6pt}{1pt}\hfill}{\labelwidth=15pt%
\labelsep=5pt \leftmargin=20pt \topsep=3pt%
\setlength{\listparindent}{\saveparindent}%
\setlength{\parsep}{\saveparskip}%
\setlength{\itemsep}{0pt} }}{\end{list}}

%----------Header--------------------%
\def\course{{\sc Fundamentos y Aplicaciones de Blockchains}}
\def\term{Depto. de Computaci\'{o}n, UBA, 2do. Cuatrimestre 2025}
\def\prof{Lecturer: Juan Garay}
\newcommand{\handout}[5]{
   \renewcommand{\thepage}{\arabic{page}}
   \begin{center}
   \framebox{
      \vbox{
    \hbox to 5.78in { \hfill \large{\course} \hfill }
    \vspace{2mm}
%    \hbox to 5.78in { \hfill \large{\prof} \hfill }
%       \vspace{2mm}
       \hbox to 5.78in { {\Large \hfill \textbf{#5}  \hfill} }
       \vspace{2mm}
       \hbox to 5.78in { \term \hfill \emph{#2}}
       \hbox to 5.78in { {#3 \hfill \emph{#4}}}
      }
   }
   \end{center}
   \vspace*{4mm}
}
\newcommand{\hw}[4]{\handout{#1}{#2}{#3}{#4}{Homework #1}}

% -- For ignoring stuff -- %
\newcommand{\ignore}[1]{}

\begin{document}

%----Specs: change accordingly-----%
\newif\ifstudent % comment out false
\studenttrue 
% \studentfalse

\def\hwnum{4}
\def\issuedate{1/11/25}
\def\duedate{13/11/25, 17:00 hs} % 
\def\yourname{Juan DElia} % put your name here
%------------------------------%

\ifstudent
\hw{\hwnum}{\issuedate}{Student: \yourname}{Due: \duedate}%
\else
\hw{\hwnum}{\issuedate}{\prof}{Due: \duedate}%
\fi

% \ignore{

\noindent \textbf{Instructions}

\begin{itemize}
    \item Upload your solution to Campus; make sure it's only one file, and clearly write your name on the first page. Name the file \textsf{`$<$your last name$>$\_HW4.pdf.'} 
    %{\bf Important:} Make sure to tap {\bf Turn in} after you upload your solution.   
      
     If you are proficient with \LaTeX, you may also typeset your submission and submit in PDF format. To do so, uncomment the ``\%\textbackslash begin\{solution\}'' and ``\%\textbackslash end\{solution\}'' lines and write your solution between those two command lines.
    
      \item Your solutions will be graded on \emph{correctness} and
    \emph{clarity}. You should only submit work that you believe to be
    correct.
    % , and you will get significantly more partial credit if you     clearly identify the gap(s) in your solution.
    
    \item You may collaborate with others on this problem set.  However,
    you must \textbf{{write up your own solutions}} and \textbf{{list
      your collaborators and any external sources (including ChatGPT and similar generative AI chatbots)}} for each
    problem. Be ready to explain your solutions orally to a member of the course staff
    if asked.
    
    \ignore{
    \item For problems that require you to provide an algorithm, you must
    give a precise description of the algorithm, together with a proof
    of correctness and an analysis of its running time. You may use
    algorithms from class as subroutines. You may also use any facts
    that we proved in class or from the book.
    } %IGNORE
    
\end{itemize}

\noindent This homework contains \numquestions\ questions,
% \numpages\ pages
for a total of \numpoints \ points.
% and \numbonuspoints\ bonus points.

%\medskip
\newpage

%} %IGNORE

\begin{questions}

\question \textbf{Bitcoin backbone: Chains of variable difficulty.}
      \begin{parts}
  \part[5] Describe Bahack's ``difficulty raising'' attack.
  
    \begin{solution} %Uncomment and type your solution here
        El ataque consta de lo siguiente:
        \begin{itemize}
          \item{Suponer que en una ronda r todos los participantes tienen una cadena de longitud $\lambda m$}
          \item{
            El atacante construye una época entera por su cuenta con Timestamps falsos, lo que resulta en una dificultad muy alta para la siguiente época.
            }
          \item{Setear un $T'$ chico, tal que si computa el primer bloque de la siguiente época rápido,
          obtiene la cadena más pesada}
        \end{itemize}
        Este ataque funciona con probabilidad constante.
    \end{solution}

  \part[5] What aspect of Bitcoin's target recalculation function makes the attack ineffective? Elaborate.
  
    
    \begin{solution} %Uncomment and type your solution here
        La función de recalculo define límites en la dificultad que se puede setear en cada época.
        Esto hace que el atacante no pueda setear un T arbitrario, si no que se va acotar. Esto basta para que el atacante
        no pueda definir una cadena tan pesada como para que sea la que se adopta
    \end{solution}

\end{parts}

\newpage

\question \textbf{Proofs of Stake (PoS).}
      \begin{parts}
      \part[5] Consider the \textit{long range attack} on a PoS blockchain in a permissionless environment. Assume that an honest-stake majority always holds, and that all the parties are aware of the global clock. Describe the scenarios wherein the honest parties lose their advantages, and explain why freshly joining parties cannot distinguish an honest chain from other chains by the longest-chain selection rule.
      
    \begin{solution} %Uncomment and type your solution here
        Como PoS es un recurso virual es "simulable" a bajo costo, a diferncia de PoW que hay que gastar recursis fisicos.
        Por esto es posible construir una cadena alternativa, sin ningun costo sustancial. 
        
        Como  para el consenso se usa la regla de la cadena mas larga esto significa un problema.
        
        Si una "partie" se une por primera vez, o no se conecta durante mucho tiempo no tiene manera de distinguir estas
        cadenas ficticias con las reales.
    \end{solution}
    
    \part[5] A PoS protocol performs leader election based on the stakes each party owns. Recall that the initial stake distribution is hard-coded in the genesis block. Does a PoS protocol assume a PKI? Further, the stakes in a party's account have monetary values. Where can the initial stakes come from?
        
    \begin{solution} %Uncomment and type your solution here
        Si asume una forma de PKI. El protocolo hace una eleccion segun el stake de cada party. 
        Ese party demuestra que es el dueño de ese stake con un esquema de PKI, su public key esta asociada a ese stake.

        Como en el bloque genesis se define la distribucion inicial de stake, se podria hacer algun tipo de "venta" de ese stake. 
        Una vez que partys tienen stake pueden empezar a ser elegibles para ser lideres y  crear bloques nuevos. A su vez 
        ellos podrian vender stake a otros patys y que la red se amplie.

        El stake podria seguir expnandiendose por ejemplo recompensando a los lideres que crean bloques validos con stake nuevo.
    \end{solution}

\end{parts}

\newpage

\question  \textbf{Verifiable Random Functions (VRFs).} 
\begin{parts}

\part[4] Enumerate the security properties a VRF should satisfy. 

    \begin{solution} %Uncomment and type your solution here
      \begin{itemize}
        \item{El resultado de una VRF dado un input X debe ser Pseudo aleatorio}
        \item{Para un par (VK,SK) y un input X el output debe ser unico}
        \item{Solo quien tiene la private key puede computar el hash, pero cualquier con la public key puede ver la correctitud del hash}
      \end{itemize}    
    \end{solution}
    
\part[6] We stated in class that given a hash function, modeled as a \textit{random oracle}, and an unforgeable signature scheme, VRFs are readily realizable.  Show that that's indeed the case by proposing a construction and arguing its security---i.e., it achieves the desired security properties. Follow the VRF terminology we used in class.

    \begin{solution} %Uncomment and type your solution here
        Tomemos:
        \begin{itemize}
          \item{Keygen(r) $\rightarrow$ (VK,SK) que genera el par de claves par al VRF. Por hipotesis tenemos un esquema
          de firmas no falsificables}
          \item{Eval(SK,X) $\rightarrow$ (Y,$\Pi$) Que dado un input X genera el string Y que es pseudorandom y una prueba $\Pi$. 
          Por hipotesis tenemos una funcion de Hash modelada como un oraculo aleatorio}
          \item{Este algoritmo es ejecutado por cualquier para verificar que Y es el resultado correcto  para X con la clave pública VK y la prueba.}
        \end{itemize}

        Veamos que esto es suficiente:
        
        \begin{itemize}
          \item{El resultado es PseudoAleatorio: Como por hipotesis tenemos un oraculo aleatorio y SK es secreto, nadie que conozca
          X puede conocer el valor de Y. Ese valor de Y modela un valor aleatorio para el mundo exterior ya que no hay forma
          de con el "reconstruir" x}
          \item{El resultado es unico: Para un SK y X fijo el resultado va a ser unico porque tenemos un oraculo aleatorio}
          \item{El resultado es verificable: El esquema de firmas fue generado con un esquema de firmas no falsificables, por lo que
          los resultados son tanto verificables como seguros. Ya que cada output puede verificarse correcto mediante $\Pi$ y
          y como no se pueden falsificar las firmas es seguro.
          }
        \end{itemize}
    \end{solution}

\end{parts}

\newpage

\question \textbf{The {\em Ouroboros} protocol.}
         \begin{parts}
         
         \part[5] We saw in class that in the Ouroboros protocol the slot leader election process is abstracted out and modeled as an ``ideal functionality.'' Describe one realistic approach to elect the slot leader, and explain why there are $\bot$ symbols in the characteristic strings.
         
          \begin{solution} %Uncomment and type your solution here
            En la red el tiempo esta dividido en rondas llamadas slots. El simbolo $\bot$ es un slot vacio.

            ....
          \end{solution}
   
        \part[5] Describe the implementation of the {\em dynamic} stake setting. Why can't the parties use the most recent stake distribution (i.e., the stake distribution at the end of the previous epoch)?

    %\begin{solution} %Uncomment and type your solution here
        
    %\end{solution}
    
    \end{parts}

\newpage

~\\

\end{questions}

\end{document}
