\documentclass[12pt,addpoints,answers]{exam}
\usepackage[utf8]{inputenc}
\usepackage{amsmath,amsfonts,amssymb,amsthm}
\usepackage[margin=1in]{geometry}
\usepackage{mathtools}
\usepackage{hyperref}
\usepackage{fullpage}
\usepackage{microtype}
\usepackage{xspace}
\usepackage[svgnames]{xcolor}
\usepackage[sc]{mathpazo}
\usepackage{enumitem}
\usepackage{bm}

\pagestyle{head}

% TIRET DEFINITIONS

\newlength{\saveparindent}
\setlength{\saveparindent}{\parindent}
\newlength{\saveparskip}
\setlength{\saveparskip}{\parskip}
\newcounter{ctr}
\newcounter{savectr}
\newcounter{ectr}
\newcounter{sctr}


\newenvironment{tiret}{%
\begin{list}{\hspace{2pt}\rule[0.5ex]{6pt}{1pt}\hfill}{\labelwidth=15pt%
\labelsep=5pt \leftmargin=20pt \topsep=3pt%
\setlength{\listparindent}{\saveparindent}%
\setlength{\parsep}{\saveparskip}%
\setlength{\itemsep}{0pt} }}{\end{list}}

%----------Header--------------------%
\def\course{{\sc Fundamentos y Aplicaciones de Blockchains}}
\def\term{Depto. de Computaci\'{o}n, UBA, 2do. Cuatrimestre 2025}
\def\prof{Lecturer: Juan Garay}
\newcommand{\handout}[5]{
   \renewcommand{\thepage}{\arabic{page}}
   \begin{center}
   \framebox{
      \vbox{
    \hbox to 5.78in { \hfill \large{\course} \hfill }
    \vspace{2mm}
%    \hbox to 5.78in { \hfill \large{\prof} \hfill }
%       \vspace{2mm}
       \hbox to 5.78in { {\Large \hfill \textbf{#5}  \hfill} }
       \vspace{2mm}
       \hbox to 5.78in { \term \hfill \emph{#2}}
       \hbox to 5.78in { {#3 \hfill \emph{#4}}}
      }
   }
   \end{center}
   \vspace*{4mm}
}
\newcommand{\hw}[4]{\handout{#1}{#2}{#3}{#4}{Homework #1}}

% -- For ignoring stuff -- %
\newcommand{\ignore}[1]{}

\begin{document}

%----Specs: change accordingly-----%
\newif\ifstudent % comment out false
\studenttrue 
% \studentfalse

\def\hwnum{3}
\def\issuedate{9/10/25}
\def\duedate{21/10/25, 15:00 hs} % 
\def\yourname{Juan DElia} % put your name here
%------------------------------%

\ifstudent
\hw{\hwnum}{\issuedate}{Student: \yourname}{Due: \duedate}%
\else
\hw{\hwnum}{\issuedate}{\prof}{Due: \duedate}%
\fi

% \ignore{

\noindent \textbf{Instructions}

\begin{itemize}
    \item Upload your solution to Campus; make sure it's only one file, and clearly write your name on the first page. Name the file \textsf{`$<$your last name$>$\_HW3.pdf.'} 
    %{\bf Important:} Make sure to tap {\bf Turn in} after you upload your solution.   
      
     If you are proficient with \LaTeX, you may also typeset your submission and submit in PDF format. To do so, uncomment the ``\%\textbackslash begin\{solution\}'' and ``\%\textbackslash end\{solution\}'' lines and write your solution between those two command lines.
    
      \item Your solutions will be graded on \emph{correctness} and
    \emph{clarity}. You should only submit work that you believe to be
    correct.
    % , and you will get significantly more partial credit if you     clearly identify the gap(s) in your solution.
    
    \item You may collaborate with others on this problem set.  However,
    you must \textbf{{write up your own solutions}} and \textbf{{list
      your collaborators and any external sources (including ChatGPT and similar generative AI chatbots)}} for each
    problem. Be ready to explain your solutions orally to a member of the course staff
    if asked.
    
    \ignore{
    \item For problems that require you to provide an algorithm, you must
    give a precise description of the algorithm, together with a proof
    of correctness and an analysis of its running time. You may use
    algorithms from class as subroutines. You may also use any facts
    that we proved in class or from the book.
    } %IGNORE
    
\end{itemize}

\noindent This homework contains \numquestions\ questions,
% \numpages\ pages
for a total of \numpoints \ points.
% and \numbonuspoints\ bonus points.

%\medskip
\newpage

%} %IGNORE

\begin{questions}


\question \textbf{Bitcoin transactions:}
         \begin{parts}
                   \part[7] Describe the mechanism used in the actual Bitcoin application that  the miners follow in order to insert transactions into a block. Would the Consistency (aka Persistence) and Liveness properties we saw in class, as well as the $V,I,R$ functions have to be modified to capture the actual mechanism? Elaborate.
                   
     \begin{solution} %Uncomment and type your solution here
        Para describir el mecanismo que usan los mineros de Bitcoin para agregar un bloque use los siguientes recursos:

        \href{https://bitcoin.org/bitcoin.pdf}{En la seccion 5 del paper}

        \href{https://developer.bitcoin.org/devguide/mining.html}{El sitio de developers de Bitcoin}

        El mecanismo se resume en:

        \begin{enumerate}
            \item{Nuevas transacciones son transmitidas en la red a los nodos.}
            \item{Cada nodo recolecta transacciones para formar un bloque, sumando la public key a donde se debe enviar el coinbase.}
            \item{Cada nodo intenta resolver el PoW para el target correspondiente}
            \item{Cuando un nodo encuentra una PoW transmite el bloque en la red para todos los nodos}
            \item{Los nodos de la red aceptan el bloque si todas las transacciones son validas usando las merkle root y verificando
            que no haya double spending}
            \item{Los nodos expresan su aceptacion de un bloque poniendose a crear el siguiente bloque de la cadena, usando
            como "hash del bloque anterior" el hash del bloque que aceptan}
        \end{enumerate}

        No es necesario cambiar Consistency (Everyone sees the same history) ni Liveness (Everyone can add new transactions). Ambas
        tienen sentido en el contexto del mecanismo real de bitcoin. Por como funciona el mecanismo la idea es que mediante PoW 
        se mantenga la persistencia de las transacciones, ya que se extienden las cadenas donde se usa "esfuerzo" si la mayoria
        de los nodos son honestos todos van a ver la misma cadena, o al menos el mismo prefijo (los ultimos bloques son mas volatiles).

        Tampoco hace falta cambiar Liveness (Everyone can add new transactions). Cualqueira puede emitir transacciones en la red.
        
        No nos interesa cambiar la idea de las funciones. Por como estan definidos en el paper seccion 3.1 sirven
        para expresar la esencia del mecanismo:

        \begin{itemize}
            \item{V: The content validation predicate receives as input the
            content of a chain C, denoted by xC, and will return 1 if and only if the contents are consistent
            with the intended application implemented on top of the chain.
            
            Esto nos intteresa hacerlo, lo hacen los nodos de la red cuando verifican que todas la transacciones de la cadena
            sean validas  con las merklee root y que no haya double spending.
            }
            \item{I: It receives as input a tuple, (st, C, round, Input(), Receive()),
            that stands respectively for state data st, current chain C, current round round, contents of
            input tape Input() and contents of network tape Receive(). Given these, it will produce
            an updated state st' as well as an input x that should be the next input to be inserted in a
            block. For instance, I(·) can be as simple as copying the contents of the input tape into x
            and keeping st = $\epsilon$ , or performing a more complex operation that involves parsing C or even
            maintaining old input values that have not yet been processed as part of the state st.}

            \item{R: It receives as input a chain C and provides an interpretation of it. 
            In the simplest case it can be just returning xC and leaving it to the callee to process the contents of the chain.
            
            Nos interesa, en la aplicacion de blockchain cumple el proposito de listar las transacciones de los bloques.}
        \end{itemize}

     \end{solution}

            \part[3] Describe the purpose of a {\em coinbase} transaction. 
            
    \begin{solution} %Uncomment and type your solution here

        La transaccion coinbase es la primera de cada bloque y representa la recompensa que se le va
        a pagar a quien mine el bloque. Esta transaccion crea bitcoins, sirve no solo como recompensa sino 
        como forma de poner moneda en circulacion, ya que no hay una autoridad central que lo haga.


    \end{solution}

            \end{parts}


\newpage

\question \textbf{Proofs of work:}

\begin{parts}

\part[5] Describe the purpose and implementation of the {\bf 2x1 PoW} technique.

     \begin{solution} %Uncomment and type your solution here
        La idea de implementar 2x1 PoW es lograr consenso soportando $\frac{1}{2}$ de nodos deshonestos, a diferencia de
        las soluciones que vimos que toleraban $\frac{1}{3}$.
        
        Conceptualmente el problema es que el "backbone protocol" no asegura sufiente chain-quality. No sabemos si 
        suficientes bloques provienen de nodos honestos.
        
        Para garantizar que la cantidad de transacciones sea equivalente al "hashing power" de cada uno se puede implementar
        PoW para "minar" transacciones. Va a haber un doble proceso de mineria, uno para generar bloques y otro para 
        insertar transacciones. Asi soporta hasta un poder de hashing menor a $\frac{1}{2}$ de los participantes maliciosos.


        En lineas generales se divide en dos etapas:
        
        \begin{itemize}
            
            \item{Operation: En cada ronda las "parties" corren dos protocolos en paralelo. El primero es $\Pi_{PL}$ 
            que mantiene el ledger de transacciones y requiere q queries al oraculo $H_0(.)$. El segundo proceso
            es el de "transaction production $\Pi_{tx}$" que continuamente genera transacciones que satisfecen
            $(H_1(ctr,G(nonce,v)) < T ) \land (ctr \leq q)$ El protocolo hace q queries a $H_1(.)$}
            
            \item{Termination: Despues de la ronda L, una partie recolecta todas las transacciones unicas que hay
            en los primeros $O(k)$ bloques y retorna el valor que mas aparece de entre las transacciones}

        \end{itemize}

        El problema de tener dos subprocesos es que un adversario podria mover su poder de computo de uno al otro.

        Este protocolo que usa dos subprocesos de PoW se puede juntar en un unico proceso que usa un unico oraculo
        $H(.)$ con un total de q queries por ronda. Se puede conseguir cambiando la estructura de los PoWs de pares
        de la pinta $\langle w, ctr\rangle$ a $\langle w,ctr,label \rangle$

        A grandes rasgos es como el algoritmo que conocemos de PoW pero en cada iteracion verifica si resolvio una PoW y/o la otra
        para un mismo target. Cuando verifica si H(ctr,h) es menor al Target lo hace tanto para ese hash como para el reverso. De esta manera verifica
        dos PoW \textbf{independientes} en un unico intento.

        (El algoritmo esta en la pagina 36 de The Bitcoin Backbone Protocol: Analysis and Applications)
     
    \end{solution}

\part[7] Let $T_1$ and $T_2$ be the target values in 2x1 PoW, and $\kappa$ the size of the hash function's output. What relation should they satisfy for the technique to work? Elaborate.

    \begin{solution} %Uncomment and type your solution here
        Tiene que valer que $T_1 = T_2 = T$ y $t > k/2$ tomando:

        Sea $U$ una variable aleatoria uniforme en el rango sobre los enteros $[0, 2^k)$ y $T$ un entero tal que: $T = 2^t$.
        Para $t > k/2$ los eventos $U < T$ y $[U]^R < T$ son independientes y ambos tienen probabilidad $T \cdot 2^{-k}$  

        U y su reverso son los hashes que explique en el inciso anterior que buscamos para el mismo Target

        Dem:

        Se puede ver que cada evento sucede con probabilidad $T \cdot 2^{-k}$. La conjuncion de los eventos es
        elegir un entero U tal que $U < T$ y $U^R < T$. Como $T = 2^t$ la probabilidad condicional de $U < T$ deja 
        los t bits menos significativos de U de manera unifomemente aleatoria y fija el resto de los $\kappa - t$ bits.

        El caso para  $U^R$ es análogo. Quedan los t bits mas significativos de manera uniformemente aleatoria y los $\kappa - t$
        restantes quedan fijos.

        El evento $U^R < T$ tiene probabilidad $2^{t-(k-t) / 2^t} = T \cdot 2^{-k}$ en el espacio condicional, asi
        que ambos eventos son independientes.


    \end{solution}
    
  \part[8] Design and argue correctness of an $\ell$x1 PoW scheme. Note that such a scheme would enable an $\ell$-parallel blockchain. What properties of the blockchain or ledger application would, if any,  benefit from a parallel blockchain? Elaborate.
  
   \begin{solution} %Uncomment and type your solution here
        \textbf{CONSULTAR}

        Idea: l tiene que ser multiplo de 2. De esa manera se pueden correr l/2 blockchains. (tengo l/2 ledgers???).
        Cada uno creo que que mantiene las propiedades que tendria una individual, consistency, liveness.
        
        En una sola rona se podrian minar hasta l/2 bloques por ronda asi que se beneficiaria la propiedad de chain growth.
        Como cada cadena individual cumple las propiedades originales es un esquema correcto teoricamente.

        Problemas?: como mantengo independencia entre cada una?
    \end{solution}
    
    \end{parts}
    
\newpage

\question \textbf{Strong consensus:} We saw how the Bitcoin backbone protocol can be used to solve the {\em consensus} problem (aka {\em Byzantine agreement}). In the {\em strong consensus} problem, the {\em Validity} condition is strengthened to require that the output value be one of the honest parties' inputs---this property is called {\em Strong Validity}. (Note that this distinction is relevant only in the case of non-binary inputs.)

\begin{parts}

\part[5] What should be the assumption on the adversarial computational power (similar to the ``Honest Majority Assumption'') for Strong Validity to hold?

    \begin{solution} %Uncomment and type your solution here
        Pagina 37 del paper
    \end{solution}

\part[5]  State and prove the Strong Validity lemma.

    %\begin{solution} %Uncomment and type your solution here
        
    %\end{solution}

\end{parts}

\newpage

\question \textbf{Smart contract programming:}  In this assignment you will create your {\bf own custom token}. Your contract should implement the public API described below:

\begin{itemize}
\item \underline{\textbf{owner:}} a public payable address that defines the contract’s ``owner,'' that is, the user that deploys the contract
\item \textbf{\textit{Transfer(address indexed from, address indexed to, uint256 value):}} an event that contains two addresses and a uint256
\item \textbf{\textit{Mint(address indexed to, uint256 value):}} an event that contains an address and a uint256
\item \textbf{\textit{Sell(address indexed from, uint256 value):}} an event that contains an address and a uint256

\item \textbf{totalSupply():} a view function that returns a uint256 of the total amount of minted tokens
\item \textbf{balanceOf(address\_account):} a view function returns a uint256 of the amount of tokens
an address owns
\item \textbf{getName():} a view function that returns a string with the token’s name
\item \textbf{getSymbol():} a view function that returns a string with the token’s symbol
\item \textbf{getPrice():} a view function that returns a uint128 with the token’s price (at which users can
redeem their tokens)

\item \textbf{transfer(address to, uint256 value):} a function that transfers \emph{value} amount of tokens
between the caller’s address and the address to; if the transfer completes successfully, the
function emits an event \emph{Transfer} with the sender’s and receiver’s addresses and the amount
of transferred tokens and returns a boolean value (\emph{true})
\item \textbf{mint(address to, uint256 value):} a function that enables \emph{only the owner} to create value new tokens and give them to address to; if the operation completes successfully, the
function emits an event \emph{Mint} with the receiver’s address and the amount of minted tokens
and returns a boolean value (\emph{true})
\item \textbf{sell(uint256 value):} a function that enables a user to sell tokens for wei at a price of \emph{600 wei per token}; if the operation completes successfully, the sold tokens are removed from
the circulating supply, and the function emits an event \emph{Sell} with the seller’s address and the
amount of sold tokens and returns a boolean value (\emph{true})
\item \textbf{close():} a function that enables \emph{only the owner} to destroy the contract; the contract’s
balance in wei, at the moment of destruction, should be transferred to the owner's address
\item \textbf{fallback} functions that enable anyone to send Ether to the contract’s account
\item \textbf{constructor} function that initializes the contract as needed

\end{itemize}
      
You should implement the smart contract and deploy it on the course's Sepolia Testnet. Your contract should be as secure and gas efficient as possible. After deploying your contract, you should buy, transfer, and sell a token in the contract. 

Your contract should implement the above API \textbf{exactly as specified}. \emph{Do not} omit implementing one of the above variables/functions/events, do not change their name or parameters, and do not add other public variables/functions. You can define other private/internal functions/variables, if necessary.

You should provide:

\begin{parts}

\part[5] A detailed description of your high-level design decisions, including (but not limited to):
\begin{tiret}
\item  What internal variables did you use?
\item What is the process of buying/selling tokens and changing the price?
\item How can users access their token balance?
%\item How did you link the library to your contract?
\end{tiret}

    %\begin{solution} %Uncomment and type your solution here
        
    %\end{solution}

\part[5] A detailed gas evaluation of your implementation, including:
\begin{tiret}
\item The cost of deploying and interacting with your contract.
\item Techniques to make your contract more cost effective.
%\item What was the gas impact of using the deployed library instance, compared to including its code in your contract?
\end{tiret}

    %\begin{solution} %Uncomment and type your solution here
        
    %\end{solution}

\part[5] A thorough listing of potential hazards and vulnerabilities that can occur in the smart
contract and a detailed analysis of the security mechanisms that can mitigate these
hazards.

    %\begin{solution} %Uncomment and type your solution here
        
    %\end{solution}

\part[5] The transaction history of the deployment of and interaction with your contract.

    %\begin{solution} %Uncomment and type your solution here
        
    %\end{solution}

\part[5] The code of your contract.

    %\begin{solution} %Uncomment and type your solution here
        
    %\end{solution}

\end{parts}

\newpage


~\\

\end{questions}

\end{document}
